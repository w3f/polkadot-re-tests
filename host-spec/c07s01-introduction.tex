\section{Introduction}

Validators are responsible for guaranteeing the validity and availability of PoV
blocks. There are two phases of validation that takes place in the AnV protocol.
\newline

The primary validation check is carried out by parachain validators who are
assigned to the parachain which has produced the PoV block as described in
Section \ref{sect-primary-validation}. Once parachain validators have validated
a parachain's PoV block successfully, they have to announce that according to
the procedure described in Section \ref{sect-primary-validaty-announcement}
where they generate a statement that includes the parachain header with the new
state root and the XCMP message root. This candidate receipt and attestations,
which carries signatures from other parachain validators is put on the relay
chain.
\newline

As soon as the proposal of a PoV block is on-chain, the parachain validators
break the PoV block into erasure-coded chunks as described in Section
\ref{defn-erasure-coded-chunks} and distribute them among all validators. See
Section \ref{sect-distribute-chunks} for details on how this distribution takes
place.
\newline

Once validators have received erasure-coded chunks for several PoV blocks for
the current relay chain block (that might have been proposed a couple of blocks
earlier on the relay chain), they announce that they have received the erasure
coded chunks on the relay chain by voting on the received chunks, see Section
\ref{sect-voting-on-availability} for more details.
\newline

As soon as $>2/3$ of validators have made this announcement for any parachain
block we \emph{act on} the parachain block. Acting on parachain blocks means we
update the relay chain state based on the candidate receipt and considered the
parachain block to have happened on this relay chain fork.
\newline

After a certain time, if we did not collect enough signatures approving the
availability of the parachain data associated with a certain candidate receipt
we decide this parachain block is unavailable and allow alternative blocks to be
built on its parent parachain block, see \ref{sect-unavailability}.
\newline

The secondary check described in Section \ref{sect-approval-checking}, is done
by one or more randomly assigned validators to make sure colluding parachain
validators may not get away with validating a PoV block that is invalid and not
keeping it available to avoid the possibility of being punished for the attack.
\newline

During any of the phases, if any validator announces that a parachain block is
invalid then all validators obtain the parachain block and check its validity,
see Section \ref{sect-escalation} for more details.
\newline

All validity and invalidity attestations go onto the relay chain, see Section
\ref{sect-publishing-attestations} for details. If a parachain block has been
checked at least by certain number of validators, the rest of the validators
continue with voting on that relay chain block in the GRANDPA protocol. Note
that the block might be challenged later.
\newline

\section{Preliminaries}

\begin{definition}
  \label{defn-scale-codec}
  The Polkadot project uses the \textbf{SCALE codec} to encode common data types
  such as integers, byte arrays, varying data types as well as other data
  structure. The SCALE codec is defined in a separate document known as
  \textit{"The Polkadot Host - Protocol Specification"}.
  \todo{@fabio: link to document}
\end{definition}

\begin{definition}
In the remainder of this chapter we assume that $\rho$ is a Polkadot Parachain
and $B$ is a block which has been produced by $\rho$ and is supposed to be
approved to be $\rho$'s next block. By $R_{\rho}$ we refer to the
\textbf{validation code} of parachain $\rho$ as a WASM blob, which is
responsible for validating the corresponding Parachain's blocks.
\end{definition}

\begin{definition}
  \label{defn-witness-proof}
  The {\b witness proof} of block $B$, denoted by {\bf $\pi_B$}, is the set of
  all the external data which has gathered while the $\rho$ runtime executes
  block $B$. The data suffices to re-execute $R_{\rho}$ against $B$ and achieve
  the final state indicated in the $H(B)$.
\end{definition}

This witness proof consists of light client proofs of state data that are
generally Merkle proofs for the parachain state trie.  We need this because
validators do not have access to the parachain state, but only have the state
root of it.

\begin{definition}
  \label{defn-pov-block}
  Accordingly we define the {\bf proof of validity block} or {\bf PoV} block in
  short, {\bf $\PoVB$}, to be the tuple:
  \[
  (B, \pi_B)
  \]
  A PoV block is an extracted Merkle subtree, attached to the block.
  \todo{@fabio: clarif this}
\end{definition}

\subsection{Extra Validation Data}

Validators must submit extra validation data to Runtime $R_{\rho}$ in order to
build candidates, to fully validate those and to vote on their availability.
Depending on the context, different types of information must be used.
\newline

Parachain validators get this extra validation data from the current relay
chain state. Note that a PoV block can be paired with different extra
validation data depending on when and which relay chain fork it is included
in. Future validators would need this extra validation data because since the
candidate receipt as defined in Definition \ref{defn-candidate-receipt} was
included on the relay chain the needed relay chain state may have changed.

\begin{definition}
  \label{defn-upgrade-indicator}
  $R^{up}_{\rho}$ is an varying data type (Def. \ref{defn-scale-codec}) which
  implies whether the parachain is allowed to upgrade its validation code.
  \[
    R^{up}_{\rho} := Option(H_i(B^{relay}_{chain})+n)
  \]

  \todo{@fabio: adjust formula?}

  If this is $Some$, it contains the number of the minimum relay chain height
  at which the upgrade will be applied, assuming an upgrade is currently
  signaled \todo{@fabio: where is this signaled?}. A parachain should enact its
  side of the upgrade at the end of the first parachain block executing in the
  context of a relay-chain block with at least this height. This may be equal to
  the current perceived relay-chain block height, in which case the code upgrade
  should be applied at the end of the signaling block.
\end{definition}

\begin{definition}  \label{defn-extra-validation-data}
  The \textbf{validation parameters}, $v^{VP}_B$, is an extra input to the validation
  function, i.e. additional data from the relay chain state that is needed.
  It's a tuple of the following format:
  \[
  vp_B := (B, head(B_p), v^{GVS}_B, R^{up}_{\rho})
  \]

  where each value represents:

  \begin{itemize}
    \item $B$: the parachain block itself.
    \item $head(B_p)$: the parent head data (Def. \ref{defn-head-data}) of block
    $B$.
    \item $v^{GVP}_B$: the global validation parameters
    (\ref{defn-global-validation-parameters}).
    \item $R^{up}_{\rho}$: implies whether the parachain is allowed to upgrade
    its validation code (Def. \ref{defn-upgrade-indicator}).
  \end{itemize}
\end{definition}

\begin{definition}
  \label{defn-global-validation-parameters}
  The \textbf{global validation parameters}, $v^{GVP}_B$, defines global data 
  that apply to all candidates in a block.
  \[
    v^{GVS}_B := (Max^{R}_{size}, Max^{head}_{size}, H_i(B^{relay}_{chain}))
  \]

  where each value represents:

  \begin{itemize}
    \item $Max^{R}_{size}$: the maximum amount of bytes of the parachain Wasm
    code permitted.
    \item $Max^{head}_{size}$: the maximum amount of bytes of the head data
    (Def. \ref{defn-head-data}) permitted.
    \item $H_i(B^{relay}_{chain})$: the relay chain block number this is in the
    context of.
  \end{itemize}
\end{definition}

\begin{definition}
  \label{defn-local-validation-parameters}
  The \textbf{local validation parameters}, $v^{LVP}_B$, defines parachain-specific
  data required to fully validate a block. It is a tuple of the following format:
  \[
    v^{LVP}_B := (head(B_p), UINT128, Blake2b(R_{\rho}), R^{up}_{\rho})
  \]

  where each value represents:
  \begin{itemize}
    \item $head(B_p)$: the parent head data (Def. \ref{defn-head-data}) of block
    $B$.
    \item $UINT128$: the balance of the parachain at the moment of validation.
    \item $Blake2b(R_{\rho})$: the Blake2b hash of the validation code used to
    execute the candidate.
    \item $R^{up}_{\rho}$: implies whether the parachain is allowed to upgrade
    its validation code (Def. \ref{defn-upgrade-indicator}).
  \end{itemize}
\end{definition}

\begin{definition}
  The \textbf{validation result}, $r_B$, is returned by the validation code $R_{\rho}$
  if the provided candidate is is valid. It is a tuple of the following format:
  \begin{alignat*}{2}
    r_B &:= (head(B), Option(P^B_{\rho}), (Msg_0, ..., Msg_n), UINT32) \\
    Msg &:= (\mathbb{O}, Enc_{SC}(b_0,.. b_n))
  \end{alignat*}

  where each value represents:
  \begin{itemize}
    \item $head(B)$: the new head data (Def. \ref{defn-head-data}) of
    block $B$.
    \item $Option(P^B_{\rho})$: a varying data (Def. \ref{defn-scale-codec})
    containing an update to the validation code that should be scheduled in the
    relay chain.
    \item $Msg$: parachain "upward messages" to the relay chain. $\mathbb{O}$
    identifies the origin of the messages and is a varying data type (Def.
    \ref{defn-scale-codec}) and can be one of the following values:
    \begin{equation*}
      \mathbb{O} =
      \begin{cases}
        0, & \text{Signed} \\
        1, & \text{Parachain} \\
        2, & \text{Root}
      \end{cases}
    \end{equation*}
    \todo{@fabio: define the concept of "origin"}

    The following SCALE encoded array, $Enc_{SC}(b_0,.. b_n)$, contains the
    raw bytes of the message which varies in size.
    \item $UINT32$: number of downward messages that were processed by the Parachain. It
    is expected that the Parachain processes them from first to last.
  \end{itemize}
\end{definition}

\begin{definition}
  \label{defn-blob}
  Accordingly we define the {\bf erasure-encoded blob} or {\bf blob} in short,
  {\bf $\bar{B}$}, to be the tuple:
  \[
  (B, \pi_B, v^{GVP}_B, v^{LVP}_B)
  \]

  where each value represents:

  \begin{itemize}
    \item $B$: the parachain block.
    \item $\pi_B$: the witness data.
    \item $v^{GVP}_B$: the global validation parameters (Def. \ref{defn-global-validation-parameters}).
    \item $v^{LVP}_B$: the local validation parameters (Def. \ref{defn-local-validation-parameters}).
  \end{itemize}
\end{definition}

Note that in the code the blob is referred to as "AvailableData".

\section{Overal process}

The Figure \ref{diag-anv-overall} demonstrates the overall process of assuring
availability and validity in Polkadot \todo{complete the Diagram}.

\begin{figure}
  \centering
  \label{diag-anv-overall}
  \includegraphics[width=1\textwidth,height=\textheight]{figures/c07s01-overview.eps}
  \caption{Overall process to acheive availability and validity in Polkadot}
\end{figure}

